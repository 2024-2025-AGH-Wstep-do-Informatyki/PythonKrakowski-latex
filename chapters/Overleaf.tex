\section{Co to jest Overleaf?}
\label{sec:overleaf}

Overleaf to internetowy edytor LaTeX, który umożliwia użytkownikom pisanie, edytowanie i współpracę nad dokumentami LaTeX bez konieczności instalowania oprogramowania na komputerze. Jest to platforma chmurowa, co oznacza, że wszystkie pliki są przechowywane online i dostępne z dowolnego miejsca z dostępem do Internetu.

\section{Główne Funkcje Overleaf}

\begin{itemize}
    \item \textbf{Współpraca w Czasie Rzeczywistym}: Overleaf pozwala wielu użytkownikom jednocześnie edytować ten sam dokument LaTeX, co ułatwia pracę zespołową i wspólne projekty naukowe.
    \item \textbf{Podgląd na Żywo}: Użytkownicy mogą natychmiastowo zobaczyć efekty swoich zmian w podglądzie PDF, co pozwala na szybkie wykrywanie i poprawianie błędów.
    \item \textbf{Szablony Dokumentów}: Overleaf oferuje szeroki wybór szablonów, które pomagają rozpocząć pracę nad różnymi rodzajami dokumentów, takimi jak artykuły naukowe, prezentacje i raporty.
    \item \textbf{Automatyczne Zarządzanie Bibliografiami}: Platforma wspiera integrację z narzędziami do zarządzania bibliografiami, takimi jak Zotero czy Mendeley, co ułatwia dodawanie cytowań i bibliografii do dokumentów.
    \item \textbf{Historia Wersji}: Overleaf zapisuje historię wersji dokumentu, dzięki czemu użytkownicy mogą śledzić zmiany i przywracać wcześniejsze wersje, jeśli zajdzie taka potrzeba.
\end{itemize}

\section{Dlaczego Warto Korzystać z Overleaf?}

Korzystanie z Overleaf ma wiele zalet:
\begin{itemize}
    \item \textbf{Dostępność i Mobilność}: Praca na Overleaf jest możliwa z dowolnego miejsca i urządzenia z dostępem do Internetu.
    \item \textbf{Bezpieczeństwo Danych}: Dokumenty są przechowywane w chmurze, co minimalizuje ryzyko ich utraty.
    \item \textbf{Integracja z Narzędziami}: Overleaf łatwo integruje się z innymi narzędziami używanymi w pracy naukowej i zawodowej.
\end{itemize}